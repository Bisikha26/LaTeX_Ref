\chapter{LITERATURE REVIEW}

A 2018 paper named “Medusa: Powered Log Storage System” demonstrates a log based storage system based on Hyperledger. The proposed system,” Medusa” is designed for log-based auditing on critical data of enterprise and government agencies. This system yields superior performance in low latency and non-real-time batch processing scenarios\cite{wang2018medusa}.


In 15 September 2020, Yang Liu, Debiao, Mohammad S. Obaidat, Neeraj Kumar, Muhammad Khurram Khan, Kim-Kwang Raymond Choo published an article on “Journal of Network and Computer Applications”. In their paper, they provided an in-depth review of existing blockchain-based identity management papers and patents published between May 2017 and January 2020\cite{liu2020blockchain}.
 

In November 2020, M. Kuperberg published a journal in IEEE titled Blockchain-Based Identity Management: A survey from the Enterprise and Ecosystem Perspective. The paper was a systematic, criteria-driven survey of the solutions and technologies for this growing field and their comparison with the capabilities of established solutions. It included an extensive set of requirements covering ecosystem aspects, end-user functionality, mobility and overhead aspects, compliance/liability, EU regulations, standardization, and integration\cite{kuperberg2019blockchain}.

BPDIMS is a 2019 paper that discusses the benefits of using blockchain to handle user centered personal data management. It uses blockchain due to its immutability and high data integrity. The user data is stored in encrypted form using symmetric-key cryptography and the encryption key of user distributed over different key keepers using threshold cryptographic methods. This increases the security of the system and decreases the likelihood of data leakage\cite{faber2019bpdims}.

A 6 October 2021, Chang Soo Sung and Joo Yeon Park published article on “Journal of Enterprise Information Management”. Their study explores the adoption of a blockchain-based identity management system using a literature review and an actual design case intended for use by the government sector\cite{sung2021understanding}. They found that blockchain-based identity management systems can significantly improve transparency, accountability, and reliability in the user control of one’s own data while reducing the time and cost needed to deliver public services, as well as increasing administrative efficiency.

 2019 paper on “Blurring the Lines between Blockchains and Database Systems: The Case of Hyperledger Fabric”, researchers identified a strong similarity of the case of Hyperledger Fabric and distributed database systems in general. These similarities are analyzed in detail and used to transition mature techniques from the context of database systems to fabric. Transaction recording is done to remove serialization conflicts and early aborts of transactions that have no chance to commit\cite{sharma2019blurring}. 

Aadhar system was launched in India in 2010 to digitize identity. It is based on the principle of openness, linearity, strong security and vendor neutrality. It uses Api based loosely coupled modules from day one, encapsulated in reusable Api driven and must have an automatic test suite. The code base should be simple and minimal to avoid inconsistent information access\cite{dalwai2014aadhaar}.

 Research ICT Africa and the Centre for Internet and Society (CIS) partnered in 2020 and 2021 to investigate, map, and report on the state of digital identity ecosystems in 10 African countries. The project looked at local, digitized ID systems in Ghana, Kenya, Lesotho, Mozambique, Nigeria, Rwanda, South Africa, Tanzania, Uganda, and Zimbabwe\cite{ngwenya2021digital}. The research took place within parameters set by an Evaluation Framework for Digital Identities, which was developed by CIS to assess the alignment of digital identity systems for compliance with international rights and data protection norms. The Framework evaluated certain aspects of the existing governance and implementation mechanisms of digital identity in their respective and unique contexts.
 
 In the paper "Digital Identity Using Hyperledger Fabric as a Private Blockchain-Based System" published on 2023, an approach model is proposed for a private blockchain-based software application that converts supported documents into a digital format using Hyperledger fabric, which is one of the popular blockchain technologies. In addition, a verification method using face and text recognition technologies is proposed to ensure the credibility of information access. The main purpose is to keep all the personal documents, such as ID, license, and birth certificate, in one place and accessible at all times, subsequently, reducing the risk of losing or misplacing them during regular activities\cite{odeh2023digital}.

 A 2022 paper on "An International Federal Hyperledger Fabric Verification Framework for Digital COVID-19 Vaccine Passport" proposes an architecture that can verify DVPs across countries and organizations and protect personal data privacy under the encryption technology and identity the control scheme provided by Hyperledger Fabric, which prevents unauthorized personnel from being protected from viewing other people’s data.This study takes Hyperledger Fabric as the proposed architecture\cite{shih2022international}.

 In a 2022 paper titled , In this paper, a novel IBPRE scheme, DOM-IBPRE scheme is proposed and demonstrated, which is achieved by combining IBPRE, blockchain and the inter planetary file system (IPFS) technology. The scheme can avoid the complexity of certificate management, enhance the security of big data storage and ameliorate the efficiency of big data sharing. In addition, a single PKG is replaced by multi trusted authority in our scheme, the problem that the malicious attack on the PKG results in the leak of the private key of each user in conventional IBPRE scheme is solved\cite{he2021efficient}. At the same time,we also assessed the security performance of the scheme by Chosen-Plaintext Attack (CPA) based on a modified DBDH problem in the standard model and simulated our scheme with Pairing Based Cryptography (PBC) library, by comparing with other PRE schemes, our scheme has the better performance in computation .

 To store digital identity data remotely with the proper security standard required for these documents, Blockchain is a suitable technique as the data stored in it is tamper-proof and immutable. Among various blockchain, hyperledger fabric outperforms other blockchain in low latency and non-real time batch processing scenarios and is comparable to centralized IT systems. This research contains noteworthy information for anyone that work in the field of blockchain or identity storage. This research will also be helpful to us for the completion of the project.   