\chapter{CONCLUSION}
\section{CONCLUSION}
% In today's digital age, managing personal data is a complex and challenging task. The amount of personal data that individuals generate and share every day has increased significantly, and the need for a secure and reliable system to store and share personal data has become more critical than ever. With the advent of blockchain technology, there has been a surge in the development of innovative solutions to address this problem. This project is designed to provide a secure and efficient way of storing and sharing personal data using blockchain technology.

% One of the main benefits of using blockchain technology to store personal data is the high level of security it provides. Blockchain technology uses cryptography to secure the data, making it virtually impossible for unauthorized parties to access or tamper with the data. Additionally, the decentralized nature of blockchain technology ensures that there is no single point of failure, making the system highly resistant to attacks.

% The mobile app developed in this project, is designed for convenience of the users. Users can store a personal data, including driving licence information, citizenship information, National Identity Card information. The app ensures that the data is stored securely and is only accessible to the user, ensuring that their privacy is protected at all times. The app contains features which allow users to share their data securely with other institutions, such as banks, to facilitate the process of opening bank accounts, or with other users, such as traffic police or bartenders to verify the users age or driving license. Instead of relying on physical documents that can be lost or stolen, users can share their data securely using the app, making the entirety of the process much more convenient and efficient.

% The project also includes an integration with a government web app developed by the same team. This integration further enhances the security and reliability of the system by enabling users to share their data with government institutions securely. The government web app ensures that the data is transmitted securely, and the use of blockchain technology ensures that the data is tamper-proof.

% In conclusion, this project provides an innovative solution to the problem of managing personal data securely. The use of blockchain technology ensures that the data is stored securely and is only accessible to the user, while the mobile app makes it easy for users to manage and share their data. The integration with government web apps further enhances the security and reliability of the system, making it a comprehensive solution for managing personal data. Overall, this project has the potential to revolutionize the way personal data is managed and shared, making it a valuable asset for individuals and institutions alike.

We've successfully achieved all of the objectives outlined when starting the project. A fully functional Parichaya mobile application and a Karmachari web application has been developed for the use of the citizens and the government officials. Along with the system requirements that had to be implemented, additional features are provided by the application as well. The user interfaces of both the applications have been made simple enough for a layman to use with ease. At the end, the smooth operation of the components in the backend, along with the client applications showed that the system's requirements, designs and implementation were handled correctly. All the deadlines have been met meaning that the schedule chosen was suitable for the project. After completing the project, following results are obtained: 
\begin{itemize}[itemsep=-4pt, topsep=-8pt]
    \item Problem of digital authentication is solved.
    \item The use of physical identity documents is removed.
    \item Decentralization of citizen's document details is achieved. 
\end{itemize}

\section{FUTURE ENHANCEMENTS}

% Enhancements: (1 to 2 Pages)

% a. Mention approaches that were not attempted, but could have been experimented with for better results

% b. Recommended a research path for future researchers that may embark on a similar research topic

\begin{itemize}
    \item Data stored in the blockchain are not encrypted currently. Techniques like threshold cryptography can be added later to the blockchain to make it more secure.
    \item This project only handles three identity documents for the time-being i.e. Citizenship Card, National Identity Card and Driving License. It can be optimized to digitize other documents to better suit the needs of a citizen.
    \item Notifications can be sent to the user when their documents have been updated or modified. 
    
\end{itemize}






