\chapter{REQUIREMENT ANALYSIS}
\section{Functional Requirements}
Functional requirements are product features or functions that developers must implement to enable users to accomplish their tasks. Generally, functional requirements describe system behavior under specific conditions. The functional requirements for our project are given below:
\begin{itemize}[itemsep=-4pt, topsep=-8pt]
    \item The web app user should be able to register documents of citizens into the blockchain.
    \item The web app user should be able to search and update the documents of the citizen. 
    \item Users should be able to view their document details in their mobile app.
    \item User should be able to share their documents to another app user through QR code.
    \item User should be able to scan the QR codes generated by third party applications through their mobile app to share their identity details.
    \item Users should be able to scan the QR codes to view the documents shared by other users.
    \item Users should be able to view a transparent log of all their app activity.
\end{itemize}

\section{Non-Functional Requirements}
Non-functional requirements are a set of specifications that describe the system's operation capabilities and constraints and attempt to improve its functionality. The non-functional requirements required for our project are given below:
\begin{itemize}[itemsep=-4pt, topsep=-8pt]
\item The applications should be user-friendly and intuitive for ease of use. 
\item The application should perform with efficient throughput and response time.
\item The system should be scalable and should accommodate a growing user base.
\item The project should provide a tamper proof and secure digital identity storage.
\item The system should be able to handle all types of identity documents issued by the government of Nepal.

% \item The project should provide a high quality digital identity management system.
\end{itemize}


\section{Feasibility Study}
\subsection{Economic Feasibility}
To determine the economic feasibility of this project, the cost associated with developing and launching the application must be evaluated. The costs associated with this project are mostly related to costs of developing the project with the utmost security and cost of maintaining the server. This project reduces the cost of any citizen by eliminating the need for physical identity documents. It also streamlines the identity verification process, reducing the time and resources required for identity verification.

\subsection{Operational Feasibility}
The system proposed has mobile application and web application with helpful user interfaces, so the end-users wouldn't have problem operating the required application with ease. Though, there are potential risks like data breach or privacy risks, the potential benefits outweighs it. The application has the potential to provide a variety of valuable services to citizens that helps to solve physical document identification and authentication. This makes the system practical for the users as it can be implemented in their day-to-day life.

\subsection{Technical Feasibility}
Development of this project required significant resources, including financial, human, and technological resources. The development team needed access to a range of hardware and software resources, including servers and cyber security tools. An inspection to whether this system can be implemented with the available tools and experts shows an absolute requirements of at least two or more servers with decent computing and storage capability, and also multiple software developers with good knowledge of Hyperledger Fabric and frameworks for developing the required web and mobile applications.

% The titled project “Parichaya” has the potential to provide various benefits like increased security, convenience and cost savings. Usage of “ Hyperledger Fabric” to store digital identity documents makes the documents tamper proof and immutable increasing security of the documents. The project is scalable and can accommodate a growing user base. The application is user-friendly and intuitive and provides secure digital identity storage. Hence, the project if properly executed can provide a high quality digital identity management system.




\section{Software Requirements}

\textbf{Hyperledger Fabric}\newline
  Hyperledger Fabric is an open-source enterprise-grade distributed ledger technology platform that provides a modular architecture designed to support various use cases requiring high degrees of confidentiality, scalability, and flexibility. Hyperledger Fabric is used in this project to set up a test network of private blockchain. A private blockchain is designed to be accessible only to a specific group of participants. Surely single point failures can be contained using any of the public blockchains made in networks like Ethereum. It would even provide greater efficiency and cost savings compared to traditional centralized databases, as it eliminates the need for intermediaries and streamlines data exchange among participants. However, our system handles with the confidential documents of the citizens, and a blockchain where anyone can join the network and participate in the consensus process would not be ideal. Rather, it would seem logical to use a blockchain where access to the network and the ability to participate in the block creation and validation is restricted to approved entities (in our case, two organizations ‘Parichaya’ and ‘Nepal Government’)\cite{manevich2019endorsement}.
One of the key features of using Hyperledger Fabric in our system is its support for private channels, which allows different groups of network participants to transact in a private and confidential manner. This means that multiple government bodies or institutes can be used as different organizations in the network. Thus, a secure environment is created where these organizations can communicate and transfer with each other in private. Hyperledger Fabric also possesses the ability to tailor the network to meet specific business needs, meaning, there is flexibility in the number of peers involved, the CA used by each organization and the number of organizations itself. So, expansion of our system to include other government bodies would not pose any problems. 
% \end{itemize}
% \begin{itemize}

\textbf{Docker Compose} \newline
Docker Compose is a tool that allows developers to define and run multi-container Docker applications. It is used to define and manage the services and dependencies of an application as a set of Docker containers, providing a way to easily configure and manage complex, multi-container applications. Since our system uses Hyperledger Fabric, it contains various different entities like peers, certificate authorities, ordering services, and databases, working together in a network, it would be better to run each of these entities in individual docker containers. The docker compose helps in easier management, connections, networking and configuration of each of these individual docker containers. Analogically thinking, each of the docker containers are like musical notes, and the docker compose is like a composer, which manages the musical notes in way to provide the required music. Consequently, Docker allows developers to package an application and all its dependencies into a single unit that can be deployed on any system running the Docker engine. So, the Parichaya Django backend has also been placed in docker containers. This makes it easier to manage and scale applications, as we can move them between different environments without having to worry about compatibility issues.\cite{bhat2022understanding}

\textbf{Flutter}\newline
A project based on the storage of citizenship, national identity, and driving license requires a secure and reliable way to store and share sensitive documents. Flutter, being a cross-platform framework, offers a cost-effective and time-efficient way to create a mobile application that works seamlessly on both Android and iOS devices. Its pre-built widgets and tools allow developers to create a beautiful and responsive user interface that is consistent across different devices\cite{tashildar2020application}. Moreover, Flutter's built-in security features, such as encryption and secure storage, ensure that sensitive information is protected from unauthorized access. This makes it a suitable option for creating a mobile application that securely stores and shares sensitive documents like citizenship, national identity, and driving license. Integrating this application with banks' identity verification systems can significantly speed up the process of opening and managing financial accounts. With Flutter, developers can create an application that allows users to securely store their government-issued documents on their devices and share them with banks when required. Overall, Flutter offers a robust and secure solution for creating a mobile application that stores and shares sensitive documents while providing a seamless user experience across different platforms.

% \end{itemize}

% \begin{itemize}
    \textbf{React JS}\newline
   Our project involves the issuance of sensitive documents like citizenship, national identity, and driving licenses through a government portal, using ReactJS can be a valuable tool for developing the front-end interface of our application. ReactJS is a popular front-end framework that allows developers to build dynamic and responsive user interfaces. With ReactJS, developers can create reusable components and build an interactive user interface that allows authorized persons of the government to issue sensitive documents securely. ReactJS also has state management features that make it easier to handle complex user interactions and manage the application's data flow.
   Moreover, ReactJS's virtual DOM allows for faster rendering and better performance, ensuring a smooth user experience. This is particularly important for applications that involve sensitive information, where the user experience should be seamless and efficient \cite{aggarwal2018modern}.
 \newpage
 \textbf{Node JS}\newline
 In Hyperledger Fabric, chaincode (smart contracts) are In Hyperledger Fabric, the business logic that defines how peer organizations interact with the ledger, is contained in a smart contract. The structure that contains the smart contract, called chaincode, is installed on the relevant peers, approved by the relevant peer organizations, and committed on the channel. After these smart contracts have been committed, client applications can be used to invoke transactions on the chaincode, via the Fabric Gateway Application. In Parichaya, chaincode is written in Node.js using the Hyperledger Fabric SDK supported for it. The gateway applications each runs a node server that provides a set of APIs that developers can use to interact with the Fabric network and create, deploy, and invoke chaincode. 
The Node.js SDK for Hyperledger Fabric also includes a client application called the Fabric CA Client. We have used this client application for managing identities and certificates for nodes and clients in the blockchain network. It enables users to create, register, and enroll identities, and to manage the associated certificates. Although, all the required logics can be written in chaincode, we  have simplified the process, dividing the complex and simple logics into two. The chaincode only contains basic read and write operations. Further, we have used Node.js among others for developing chaincode in Hyperledger Fabric for several reasons. First, it is a popular programming language with a large developer community and many available libraries and frameworks. Second, Node.js is well-suited for developing asynchronous applications, which is important in a distributed system like Hyperledger Fabric. Finally, Node.js is easy to learn and has a low barrier to entry, making it an accessible option for developers who are new to blockchain development.
In summary, Node.js is an important technology for building blockchain applications on Hyperledger Fabric. Its SDK provides the necessary APIs for developers to interact with the blockchain network, and the Fabric CA Client enables users to manage identities and certificates.

 \textbf{Django}\newline
 Django Rest Framework makes it easy to create APIs that can be consumed by a wide range of clients, including web browsers, mobile devices, and other applications. We have used django rest framework to create multiple endpoints which interacts with the Parichaya mobile application and the node servers. The purpose of Django in our app is for simplification of the complexities that comes along with the blockchain. Mostly, it helps to separate the business logic of the system with the basic read and write operations in the blockchain. The mobile users doesn’t interact with the blockchain directly but interacts with it through the endpoints created in DRF, which subsequently invokes the required functions of the chaincode. Further, additional end-user features and complex logic like authentications, websocket connections, and many more are written in Django so as to keep a distinct separative line between the business logics and the blockchain read write operations written in the chaincode.
In summary, Django Rest Framework is an important tool to easily create APIs that can be consumed by the clients. In our system, it acts as a middleman for the communication between the end users and the blockchain’s chaincode. It contains complex logics for authentications, websocket connections, and many more.




\section{Hardware Requirements}

To run and maintain the Hyperledger Fabric test network, a device with high processing prowess is required. Basic computer having RAM even up to 16 GB is not able to handle it. Moreover, in our system, the docker containers are huge in size as it contains all of the entities of the test network, which includes all of the user’s document data along with multiple images as well. Additionally, there is also a Django server which requires to be operated simultaneously. So, a device having large processing and storage capability is required for the system to work properly.
