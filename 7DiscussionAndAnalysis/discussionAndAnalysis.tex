\chapter{DISCUSSION AND ANALYSIS}

% (20\% of Report Length)

% a. Quantitatively presenting output of verification and validation procedures

% b. Comparing between theory and simulation values

% c. Comparing with state-of-the-art work performed by other authors

% d. Performing error analysis and pinpointing possible sources of error

The idea of authenticating a country’s citizen has become a norm for any of the country’s government. Most of the developed countries have developed a well defined system for the online authentication process, which acts as a base for providing government services to the citizens. However, in case of Nepal, the authentication process is well outdated, and in dire need for an upgrade. In our research, we discovered that even the trending applications of Nepal like mobile banking or other wallet applications authenticates it’s user by matching the user’s input
with their physical documents, which must be provided by the user. Furthermore, the government of Nepal, along with other institutions, still requires physical documents for identification and other formal procedures. This might pose as a big problem as it requires time and manpower to verify and maintain the information obtained. Also, the use of physical document has a problem of its own. Neither is it convenient nor is it safe. We might not always have access to the required documents as well. Yet, attempts of digitizing such documents in Nepal just lead to failure. A major cause would be the centralized storage of all the information in one particular place “Singha durbar”, which leads to a single point of failure. The government of Nepal did try to salvage this situation by providing their own mobile application “Nagarik App”, which is performing quite well, though it received a lot of criticism in the earlier stages of development. The app itself relies on the centralized information, so it is prone to single point failure as well.

Through analysis of all the findings from our research, we felt a strong requirement for a better digital ecosystem. One, which is not centralized and prone to single point failure, and is convenient for the Nepali citizens, along with being easy to use for the verification process by government officials, so as to provide governmental services. The base of which is a private and permissioned blockchain technology called “Hyperledger Fabric”. Blockchain technology is a decentralized, digital ledger that is used to record transactions across a network of computers. One of the key features of blockchain technology is that it is secure and tamper-proof, as each block in the chain contains a record of previous transactions and is linked to the previous block through a cryptographic hash. By using a blockchain-based digital identity system, personal information can be stored and verified in a tamper-proof manner, which can help prevent identity fraud and ensure that individuals have control over their personal data. Additionally, blockchain technology can enable faster and more efficient verification of identities for financial transactions, government activities, and other important processes. Overall, a blockchain-based digital identity system can help improve security, efficiency, and transparency of citizens documents in Nepal. 

Hyperledger Fabric can be used for digital identity management by creating a permissioned blockchain network that stores and manages digital identity information. The network would be composed of authorized participants, such as government agencies, financial institutions, and other organizations that need to verify the identity of individuals. This system will solve the problem of authentication regarding digital identification process by simply authenticating the input data itself with the use of blockchain technology. Hyperledger Fabric is used in this project to set up a test network of private blockchain. This test network is simulated in containers using Docker Compose. This test network is configured to contain two organizations, ’Parichaya’ and ’Nepal Government’. Parichaya represents the organization that interacts with the blockchain to provide digital identity services through the Parichaya App. Nepal Government represents the organization that interacts with the blockchain to maintain the digital record of government issued identity documents. The two organizations agree on a channel configuration to jointly collaborate on the channel. The business logic that defines how peer organizations interact with the ledger is contained in a smart contract which is inside a structure called chaincode. This basically has the functions of reading and writing each citizens documents in the blockchain. Primarily, the chaincode transactions are invoked via a client application through the Fabric Gateway Application. This system contains two of such client applications, the Parichaya web app, and the Parichaya mobile app. The web application is connected to the blockchain through the Gateway Application for ‘Nepal Government’, and  can only be used by authenticated government officials, where logs of their activities are also kept for transparency and security purposes. Even corrupted government officials can be kept away from inflicting unwanted modifications in the blockchain. Now, for the problem of physical documents, its replacement can be assured by the use of client mobile application. This app allows users to view their government documents and also allows them to share the details to other users or to some third party application. All of these services are facilitated by the Parichaya backend server, which interacts with the blockchain through the Gateway Application for ‘Parichaya’. This system will ensure safe and secure digitization of Nepal with convenient and easy to use applications.

There have been other researches on this topic which might seem similar to our system in some sense. The research papers or articles with which our system might have certain qualities in common would be “Medusa: Powered Log Storage System”,  “Journal of Network and Computer Applications”, “Blurring the Lines between Blockchains and Database Systems: The Case of Hyperledger Fabric”. Though, most of them are only similar in the use of blockchain and its network and none are trying to solve all of the exact problems that our system does. Whereas there are some applications which are similar for its authentication and storage ability to our mobile application. “Nagarik App” would be the most obvious as our mobile application might seem as an imitation to the government launched application. But, if you look at the whole system itself, the Parichaya mobile application relies on decentralized data and ease of sharing, so it would seem as a major upgrade. Although, Yoti app and Eid-me app, does solve the problem of authentication elegantly, and provides some information details to it’s users, it doesn’t rely on decentralized information, so they both have single point of failure. Furthermore, along with sharing of basic details, the Parichaya app already has some implemented features like sharing verified age as one would need in some cases, along with sharing verified driving license digitally to traffic police. Parichaya app, in fact, also provides better visual graphics of the user’s information than most of the mentioned applications.

Although, this system seems like an answer to digitalizing of Nepal with introduction of digital government documents, it does have some limitations, which can be solved with certain enhancements. Even though, blockchain technology is being used, the citizens data can’t be completely safe. As there are authorities (government officials) who can alter the data. Granted, the log of their actions is kept, and we can find the culprit, but it is only as good as the security of the authentication process of the government officials. If somehow, the authentication of the government officials are compromised, the safety of the citizen’s data cannot be guaranteed. The data stored in the blockchain is not in encrypted form as well, so malicious acts as mentioned before will be easier. There’s also currently only single web application to register users through the government.  But, this can later be modified depending upon the needs of the government. Further, the system only stores three identity documents for the time-being, i.e. Citizenship Card, National Identity Card and Driving License. This can be optimized to digitize other documents to better suit the needs of a citizen. All of the three documents are handled in different government bodies, so it would be a better fit to use individual servers for each of the documents, but in our case, only one server is being used for all types of documents. These are some of the limitations of our system, few of which can cause critical damages.

Considering the future endeavor on avenues of this system for further research based on our findings, there might be a lot of fields where this system might extend through various areas of studies and be used in different contexts. This system has potential to be upgraded into working in sectors other than just governmental works. An implemented example would be filling forms for creating accounts (for institutes like banks or colleges). Any institutes can use the “fill using Parichaya” feature, which generates a QR code for all the required information set by the institutes themselves. The mobile app user can then proceed to  scan the QR code using the Parichaya app scanner. This shows all the information that the user will be sharing, and the user can decide to either approve or deny the request. This feature will surely be handy as the form filling process will be multiple times quicker and the institutes don’t even require to spend time verifying the data. 
Further, the ability to securely store and share identity information can streamline the process of accessing financial services. This can lead to faster processing times, reduced paperwork, and increased accessibility for under-served populations. In case of fields like healthcare, storing patient identities and medical records on a blockchain can improve the efficiency and security of healthcare systems. Medical professionals can quickly and easily verify patient identities, access medical histories, and securely share medical data across different providers. For other government services, this system can streamline the process of accessing government services such as voting, social services, and tax filing. This can increase efficiency, reduce fraud, and improve accountability. As for supply chain management, the ability to securely verify the identity of suppliers and manufacturers on a blockchain can improve transparency and traceability in supply chains. This can help reduce fraud, improve quality control, and enhance consumer confidence. These are few fields in which our system can adapt and improve the current trends of handling data. Our system is naught but a concept for securely, conveniently, and efficiently handling confidential documents of user, and can be used in various fields that require it.
