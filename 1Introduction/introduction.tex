\chapter{INTRODUCTION}
% (20% of Proposal Length)
\pagenumbering{arabic}




\section{Background}
Blockchain technology is a decentralized, digital ledger that is used to record transactions across a network of computers. One of the key features of blockchain technology is that it is secure and tamper-proof, as each block in the chain contains a record of previous transactions and is linked to the previous block through a cryptographic hash. Identity management is the process of creating, maintaining, and managing identities for individuals and organizations. With the increasing need for secure, decentralized digital identities, blockchain technology has emerged as a potential solution for identity storage and management. 

Blockchain technology has the potential to provide a secure and decentralized digital identity system for individuals and organizations in Nepal as well. By using a blockchain-based digital identity system, personal information can be stored and verified in a tamper-proof manner, which can help prevent identity fraud and ensure that individuals have control over their personal data. Additionally, blockchain technology can enable faster and more efficient verification of identities for financial transactions, government activities, and other important processes. Overall, a blockchain-based digital identity system can help improve security, efficiency, and transparency in Nepal. Hyperledger Fabric can be used for digital identity management by creating a permissioned blockchain network that stores and manages digital identity information. The network would be composed of authorized participants, such as government agencies, financial institutions, and other organizations that need to verify the identity of individuals.

\section{Motivation}
One of the primary motivations of our project is to enhance security and prevent identity fraud. By storing the national identity, citizenship, and driving license information on a blockchain, the risk of fraudulent activity is reduced. This is because the data is stored in an immutable and tamper-proof ledger, making it much harder for hackers to access or alter the data.Another motivation is to provide transparency and accessibility. By using a decentralized blockchain network, the data can be accessed and verified by banks and other authorized parties quickly and efficiently. By providing a secure and decentralized way for individuals to verify their identity, the project can help underserved populations access financial services that they might not have been able to access otherwise.

\section{Problem Statement}
There is a huge problem of authenticity in today’s world. Even the mobile banking and various wallet applications in Nepal authenticates it’s user by matching the user’s input with their physical documents, which must be provided by the user. This is a waste of time and manpower resources. Another problem is the ability to handle different types of physical documents, such as driver's licenses, passports, national ID cards, etc.
The government of Nepal and other institutions require  personal documents for identifications and other formal procedures. But, carrying a copy of such sensitive documents for everywhere is not convenient or safe. We might not always have access to required documents as well. There is also high possibility of losing such personal information.
All the digitized information of the citizen of Nepal are store in one particular place "Singha durbar". The information stored is in centralized form, so if any of the central data is compromised all of our information will be on the hand of malicious actors. 

\section{Objectives}

 \begin{itemize}

    \item To improve the efficiency and speed of identity verification and authentication processes.
    \item To provide a tamper-proof and immutable record of identity information that can be used for various purposes, and to enable self-sovereign identity management.
   
\end{itemize}

% \section{Assumptions}
% \begin{itemize}
%     \item The document issued by Government Organizations is authenticated and valid.
%     \item Each citizen  has national identity number already allocated to them and every other identity documents is associated with it.
% \end{itemize}


\section{Scope of Project}

A primary assumption that the document issued by Government Organizations is authenticated and valid, and each citizen  has national identity number already allocated to them, is required. This project can be used by the citizen for secure and tamper-proof storage of their identities and personal information. This project can be used as a secure decentralized way to store and manage personal information, which can help address issues related to identity verification and fraud prevention. This project can also help to improve the access to financial and other services for individuals and businesses, particularly in rural or under-served areas.

This project can also provide secure way for individuals to control and share their personal information with various organizations, such as banks, hospitals, and schools. It can also be used by many service providing institutions which requires our personal information to provide services to us. The institution like bank requires our citizenship for creating a bank account. Similarly, offices require our certificates of qualification to provide us a job. Universities requires our grades to give us admission. Hence, all these sensitive information can be used by different institution to provide us the services. A situation might occur where we might not always have easy access to these physical documents. But these days everyone carries their smartphones wherever they go. This project allows storing these documents in a smartphone, making it more accessible and easier to share.

\section{Potential Project Applications}
Some of the potentials of our system are:\\
\textbf{Banking and financial services}\\
     The ability to securely store and share identity information on a blockchain can streamline the process of accessing financial services. This can lead to faster processing times, reduced paperwork, and increased accessibility of services for under served populations.

\textbf{Healthcare}\\
     Storing patient identities and medical records on a blockchain can improve the efficiency and security of healthcare systems. Medical professionals can quickly and easily verify patient identities, access medical histories, and securely share medical data across different providers.

\textbf{Government services}\\
    A blockchain-based identity system can streamline the process of accessing government services such as voting, social services, and tax filing. This can increase efficiency, reduce fraud, and improve accountability. 
   % \subsection{Supply chain management}
   %   The ability to securely verify the identity of suppliers and manufacturers on a blockchain can improve transparency and traceability in supply chains. This can help reduce fraud, improve quality control, and enhance consumer confidence.

\textbf{Identity verification}\\
 A blockchain-based identity system can be used to verify the identities of individuals in various industries such as transportation, education, and hospitality. This can improve security and reduce the risk of fraud.





\section{Organisation of Project Report}
The material in this project report is organised into six chapters. Chapter 1 provides  the  introduction  to our  project.  Here  we  have  discussed about  the  need  of  the project,  background  introduction,  motivation  for  the project, scopes, some assumption and the objective of the project. Chapter 2 is the literature review  of  the  project.  It  contains  the  information related to the previous work done in the field of identity storage and sharing and various research paper for blockchain. Chapter 3 is the requirement analysis report of the project where is detail study of functional and non functional requirement of the project and  feasibility study of the project.  Chapter 4 describes the requirement to run and operate the system. It includes the  overall  working principle and various requirement to run the system and how it implemented in our project. Chapter 5 contains the visual outcomes and describes the discussion and analysis of the project. The results are displayed according to different phase. Chapter 6 provides the overall summary of the project, and also describes the further improvement that can be done in the project.


