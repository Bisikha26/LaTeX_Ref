%==============================Abstract Page=================================================
\thispagestyle{plain} 
\chapter*{\vspace{0.4cm}\centerline{ABSTRACT}}
\addcontentsline{toc}{chapter}{ABSTRACT}

% Abstract (200 to 250 Words)
\vspace{-1cm}
In the current era, every field of a human’s life has been digitized to some extent. Identity documents like Rastriya Parichaya Patra (National identity card), Driving License or Citizenship card is still in primitive stage. Citizens must carry their physical documents anywhere, increasing the risk of its loss. The reason for the lack of advancement in this filed is due to lack of interest of government to adapt with time and due to problems like frauds and identity theft that may arise if authentication is not done well. The risk of data modification by malicious user is also a problem that may be occur with standard form of storage systems. 
The titled project "Parichaya: A Digital Identity Ecosystem based on Hyperledger Fabric " intends to bring the technological capability of identity documents on par with other fields by creating an ecosystem where governmental bodies and citizens can store documents and access it as per their needs with the confidence that their files are not being tampered or manipulated and is authenticated to meet the utmost standard required for sensitive and essential documents because of the use of Hyperledger which is a which is a permission based blockchain distributed system and data stored in Hyperledger is tamper-proof and immutable because each block of data is connected with other data in the order it was entered to store the information. 



\par
\textbf{Keywords:} \textit{Digital Identity, Blockchain, Hyperledger Fabric.}
%=============================================================================================
% (a) Inclusion of three to four Keywords (Lexicographical Order)
\newpage